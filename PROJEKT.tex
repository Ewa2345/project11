\documentclass{article}
\usepackage[cp1250]{inputenc}
\begin{document}
\SweaveOpts{concordance=TRUE}

\centerline{\Large{\textsc{Amortyzacja kredytu o stałych ratach i stałych ratach kapitałowych}}}
\setlength{\parskip}{4mm}





\noindent {\large\bf 1. Kredyt o stałych ratach}
\bigskip





Kredyt spłacany jest w stałych ratach wysokości $R$, na koniec każdego z kolejnych $n$ lat. Wówczas
\begin{itemize}
\item wielkość kredytu to $K=Ra_n $
\item wielkość kredytu pozostała do spłacenia po wpłaceniu k pierwszych rat
wynosi $B_k=Ra_n$ i przyjmujemy, że $n=n-k$
\item odsetki w k-tej racie wynoszą $I_k=R(1-v^{\textasciicircum}(n-k+1))$,
\item kapitał główny w k-tej racie wynosi $P_k=R(V{\textasciicircum}(n-k+1))$,
\end{itemize}
gdzie
\begin{equation}
\footnotesize{a_n=\frac{1-v^n}{i}}  
\end{equation}

\begin{equation}
\footnotesize{v=\frac{1}{1+i}}.
\end{equation}


\bigskip
\noindent {\large\bf{ 2. Kredyt o stałych ratach kapitałowych.}}
\bigskip






Kredyt spłacany jest w stałych ratach kapitałowych wysokości $P_k$, na koniec każdego z kolejnych $n$ lat. Wówczas
\begin{itemize}
\item wielkość kredytu to $K=nP_k$
\item wielkość kredytu pozostała do spłacenia po wpłaceniu $k$ pierwszych rat
wynosi 
\begin{equation}
B_k=K(1-\frac{k}{n})
\end{equation}
\item odsetki w k-tej racie wynoszą 
\begin{equation}
I_k=(1-\frac{k-1}{n})Ki
\end{equation}
\item kapitał główny w k-tej racie wynosi
\begin{equation}
P_k=\frac{K}{n}.
\end{equation}
\end{itemize}



\bigskip


\noindent{\large\textbf{ 3. Obliczanie amortyzacji kredytu}}




Kredyt wynosi 200 000 zł przy stopie procentowej $i=5\%$ i ma być on spłacany przez $20$ lat. Aby wyznaczyć amortyzację kredytu potrzebujemy obliczyć kapitał główny w k-tej racie, odsetki w k-tej racie i pozostały kredyt do spłacenia po spłaceniu k pierwszych rat. Najpierw policzymy amortyzację kredytu w ratach stałych a następnie w ratach stałych kapitałowych.


<<echo=FALSE>>=

K=200000
i=0.05
v=1/(1+i)
n=20
an=(1-v^n)/i
R=K/an
ak=NULL #a(n-1-k) =NULL
for (k in c(1:n-1)){
ak[k]=(1-v^k)/i
}
ak2=NULL
for(p in c(1:200)){
ak2[p]=(1-(1/(1+p*1/1000))^10)/p*(1/1000)  #pomocne 
}
I_k=NULL 
for (k in c(1:n)) {
I_k[k]=R*(1-v^(n-k+1))
}
P_k=NULL 
for (k in c(1:n)) {
P_k[k]=R*v^(n-k+1)
}
I_k2=NULL 
for (k in c(1:n)) {
I_k2[k]=K*i*(1-(k-1)/(n))
}
B_k=NULL
for(k in c(1:(n-1))){
B_k[k]=R*ak[n-k]
}
B2k=NULL
for(p in c(1:200)){
B2k[k]=K/((1-(v^n))/i)*(1-(1/(1+p*1/1000))^10)/(p*1/1000)
}
B_k[k+1]=0
R_k2=NULL 
for (k in c(1:n)) {
R_k2[k]=I_k2[k]+K/n
}
B_k2=NULL
for (k in c(1:n-1)) {
B_k2[k]=K*(1-k/n)
}
B_k2[k+1]=0

I_2=i*K*((n+1)/2)

I_3=NULL
for (p in c(1:200)){# całe osetkikredyt o ratach stałych kapitałowych dla zmiennej stopy procentowej
I_3[p]=(1/1000)*p*K*((n+1)/2)
}
I_4=NULL
for (p in c(1:200)){#kredyt o ratach stałych
I_4[p]=n*K/((1-(1/(1+(p*1/1000))^n))/(p*1/1000))-K
}  
I_k5=NULL
for (p in c(1:200)){ # kredyt o stały ratacj kapitałowych kata rata
for (k in c(1:n)){
I_k5[p]=K*p*1/1000*(1-(k-1)/(n))
}
}
I_k6=NULL 
for (k in c(1:n)){
for (p in c(1:200)){  
I_k6[p]=(K/((1-(1/(1+(p*1/1000))^n))/(p*1/1000)))*(1-(1/(1+(p*1/1000)))^(n-k+1))
}
}
B_k3=NULL    #dla rat stałych przy zmiennym i 
for (p in c(1:200)){  
B_k3[p]=K/((1-(v^n))/i)*(1-(1/(1+p*1/1000))^10)/(p*1/1000)
}

BK4=120000 # dla rat kapitałowych

@ 
\begin{enumerate}
\item[] { K-ta rata stała}
\end{enumerate}
@
<<echo=TRUE, results=verbatim>>=
R
@
\begin{enumerate}
\item[] {Kredyt pozostały do spłaty po spłaceniu k-pierwszych rat stałych}
\end{enumerate}
<<echo=TRUE, results=verbatim>>=
B_k
@
\begin{enumerate}
\item[] {Odestki w k-tej racie stałej}
\end{enumerate}
<<echo=TRUE, results=verbatim>>=
I_k
@
\begin{enumerate}
\item[]{ Kapitał główny w k-tej racie stałej}
\end{enumerate}
<<echo=TRUE, results=verbatim>>=
P_k
@
\begin{enumerate}
\item[]{ K-ta rata stała kapitałowa }
\end{enumerate}
<<echo=TRUE, results=verbatim>>=
R_k2
@
<<echo=FALSE>>=
@
\begin{enumerate}
\item[]{ Kapitał główny w k-tej racie stałej kapitałowej}
\end{enumerate}


<<echo=FALSE>>=
P_k2=K/n
@
<<echo=TRUE, results=verbatim>>=
P_k2
@
\begin{enumerate}
\item[]{ Kredyt pozostały do spłacenia po spłaceniu pierwszych k rat stałych kapitałowych}
\end{enumerate}
<<echo=TRUE, results=verbatim>>=
B_k2
@
\begin{enumerate}
\item[]{ Odsetki w k-tej racie stałej kapitałowej}
\end{enumerate}


<<echo=TRUE>>=
I_k2
@



\noindent{\large\textbf{}}



\bigskip
\noindent{\large\textbf{}}

\bigskip
\noindent{\large\textbf{}}

\bigskip
\noindent{\large\textbf{}}

\bigskip
\noindent{\large\textbf{}}

\noindent{\large\textbf{4. Łączne odsetki od kredytu}}
\begin{enumerate}
\item[]{ Łączne odsetki od całego kredytu biorąc pod uwagę raty stałe}
\end{enumerate}
<<echo=FALSE>>=
I=n*R-K
@
<<echo=TRUE>>=
I
@
\begin{enumerate}
\item[]{ Łączne odsetki od kredytu biąrąc pod uwagę raty stałe kapitałowe}
\end{enumerate}
<<echo=TRUE>>=
I_2
@
\noindent{\large\textbf{ Porównanie odsetek w k-tej racie stałej i w k-tej racie stałej kapitałowej}}
\bigskip

<<echo=FALSE, fig=TRUE>>=
plot(I_k,ylim=c(0,11000),main = "Wykres różnic",font.main=2,col.main="darkblue", col.lab="darkblue", ylab="Wysokość odsetek w ratach", xlab="Numer raty", type = "l", col=("red"),lwd="1",xaxt="n")
axis(1, at=c(1,5,10,15,20),labels=c("rata 1","rata 5","rata 10","rata 15","rata 20"))
legend("topright", cex=1.2,c("rata stała","rata stała kapitałowa"),fill =c("red","black"))
lines(I_k2,col="black")
@
\bigskip


\noindent{\large\textbf{}}








\bigskip

\noindent{\large\textbf{ { Porównanie kapitałów głównych w k-tych ratach  stałych i k-tych ratach stałych kapitałowych}}}
\begin{center}

<<echo=FALSE, fig=TRUE>>= 
plot(P_k,ylim=c(5800,16000),main = "Wykres różnic",font.main=2,col.main="darkblue", col.lab="darkblue", ylab="Wysokość kapitału głównego", xlab="Numer raty", type = "l", col=("red"),lwd="1",xaxt="n")
axis(1, at=c(1,5,10,15,20),labels=c("rata 1","rata 5","rata 10","rata 15","rata 20"))
legend("topleft", cex=1.2,c("rata stała","rata stała kapitałowa"),fill =c("red","green"))
abline(h=P_k2, col="green")
@
\bigskip
 
 Widzimy, że kapitał główny w ratach stałach kapitałowych jest przez cały okres taki sam, natomiast kapitał główny w ratach stałych wraz ze wzrostem lat powiększa się. 


\noindent{\large\textbf{}}
\bigskip


\noindent{\large\textbf{}}


\bigskip

\noindent{\large\textbf{}}


\bigskip
\noindent{\large\textbf{}}


\bigskip

\noindent{\large\textbf{}}


\bigskip


\noindent{\large\textbf{}}


\bigskip

\noindent{\large\textbf{ Porównanie pozostałości spłat kredytu po spłaceniu pierwszych k rat stałych i pierwszych k rat stałych kapitałowych}}
<<echo=FALSE, fig=TRUE>>=
plot(B_k,ylim=c(0,200000),main = "Wykres różnic",font.main=2,col.main="darkblue", col.lab="darkblue", ylab="Wysokość spłat kredytu", xlab="Numer raty", type = "l", col=("red"),lwd="1",xaxt="n")
axis(1, at=c(1,5,10,15,20),labels=c("rata 1","rata 5","rata 10","rata 15","rata 20"))
legend("topright", cex=1.2,c("rata stała","rata stała kapitałowa"),fill =c("red","black"))
lines(B_k2,col="black")
@

\bigskip

Zauważamy, że wysokośći rat stałych pozostałych do spłaty są dużo większe od wysokości rat stałych kapitałowych pozostałych do spłaty po k latach.
\noindent{\large\textbf{}}


\bigskip
\noindent{\large\textbf{}}


\bigskip

\noindent{\large\textbf{}}
\noindent{\large\textbf{}}


\bigskip

\noindent{\large\textbf{}}


\bigskip


\bigskip
\noindent{\large\textbf{ Wpływ stopy procentowej na łączną wysokość odsetek w racie stałej i w racie stałej kapitałowej}}
@
<<echo=FALSE, fig=TRUE>>=
plot(I_3,ylim=c(20000,650000),xlim=c(1,201),main = "Wykres różnic odsetek w zależnośći od i",font.main=2,col.main="darkblue", col.lab="darkblue", ylab="Wysokość odsetek ", xlab="Oprocentowanie", type = "l", col=("red"),lwd="1",xaxt="n")
axis(1, at=c(1,50,100,150,200),labels=c("i=0.001","i=0.05","i=0.1","i=0.15","i=0.2"))
axis(2,at=c(200000,400000,600000))
legend("topleft", cex=1.2,c("rata stała","rata stała kapitałowa"),fill =c("black","red"))
lines(I_4,col="black")
@
\bigskip

Zauważmy, że im mniejsza stopa procentowa tym wysokość odsetek jest mniejsza. Stopa procentowa  przedstawiona na wykresie jest z przedziału od 0\% do 5\% .
Gdy wartość $i$= 1\% to wysokość odsetek w ratach stałych kapitałowych jest równa 21 000zł natomiast przy stopie $i$=20\% jest ona równa 420 000zł. Biorąc pod uwagę raty stałe to odsetki są dużo większe niż odsetki w ratach stałych kapitałowych. Gdy $i$=1\% to odsetki wynoszą 21 666.26 zł. Natomiast dla stopy $i$=20\%  odsetki wynoszą 621 426.12 zł. Widzimy, że lepiej w takiej sytuacji wybrać kredyt o ratach stałych kapitałowych, gdyż odsetki są dużo mniejsze, szczególnie gdy bierzemy pod uwagę stopę procentową $i$=20\%.

\noindent{\large\textbf{}}
\bigskip
\noindent{\large\textbf{}}


\bigskip


\noindent{\large\textbf{}}
\noindent{\large\textbf{}}


\bigskip


\noindent{\large\textbf{}}

\bigskip
\noindent {\bf Wysokość pozostałości kredytu do zapłaceniu po zapłaceniu pierwszych 10 rat w zależności od i}
@
<<echo=FALSE, fig=TRUE>>=
plot(B_k3,ylim=c(50000,170000),xlim=c(1,201),main = "Wykres pozostałości kredytu do zapłaty w zależnośći od i",font.main=2,col.main="darkblue", col.lab="darkblue", ylab="Wysokość kredytu ", xlab="Oprocentowanie", type = "l", col=("red"),lwd="1",xaxt="n")
axis(1, at=c(1,50,100,150,200),labels=c("i=0.001","i=0.05","i=0.1","i=0.15","i=0.2"))
legend("topright", cex=1.2,c("rata stała","rata stała kapitałowa"),fill =c("red","black"))
abline(h=100000,col="black")
@
\bigskip

Możemy zauważyć, że pozostałości kredytu do zapłaty po 10-ciu latach w ratach stałych kapitałowych nie zależą od współczynnika $i$. Natomiast spłata kredytu po 10 latach w ratach stałych zmienia się jeżeli $i$ się zmienia. Im większe $i$ tym wielkość spłaty kredytu mniejsza.


\end{center}
\newpage
\bigskip

\bigskip


\noindent{\large\textbf{}}

\bigskip


\noindent{\large\textbf{}}

\centerline{\large{\textbf{Kredyt o stałych ratach}}}
<<echo=FALSE, results=tex>>=
library(xtable) 
lp=1:n #zaladowac biblioteke
desc <- data.frame(lp,R,P_k,B_k,I_k)
                                   row.names=c("Dane")
names(desc)<-c("Lp.","Rata","Kapitał główny","Pozostało do spłacenia","Odsetki")
desc.table <-xtable(desc, caption="Raty stałe",
                                  label="Kredyt")
print(desc.table, include.rownames=TRUE)
@

\newpage
\bigskip

\bigskip


\noindent{\large\textbf{}}

\bigskip


\noindent{\large\textbf{}}

\centerline{\large{\textbf{Kredyt o ratach stałych kapitałowych}}}
<<echo=FALSE, results=tex>>=
library(xtable) 
lp=1:n #zaladowac biblioteke
desc <- data.frame(lp,R_k2,P_k2,B_k2,I_k2)
                                   row.names=c("Dane")
names(desc)<-c("Lp.","Rata","Kapitał główny","Pozostało do spłacenia", "Odsetki")
desc.table <-xtable(desc, caption="Raty stałe kapitałowe ",
                                  label="Kredyt")
print(desc.table, include.rownames=TRUE)
@
                                                



\end{document}
